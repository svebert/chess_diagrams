\documentclass[12pt]{article}
\usepackage{amsmath}
\usepackage{booktabs}
\usepackage{geometry}
\usepackage{url}
\usepackage[utf8]{inputenc}
\usepackage[T1]{fontenc}
\usepackage{lmodern}
\usepackage{graphicx}

% Für Code-Listings
\usepackage{listings}
\usepackage{xcolor}
\lstset{
  basicstyle=\ttfamily\small,
  keywordstyle=\color{blue},
  commentstyle=\color{gray},
  stringstyle=\color{orange},
  showstringspaces=false,
  frame=single,
  breaklines=true,
  captionpos=b
}

\geometry{a4paper, margin=2.5cm}

\title{Estimating the Number of Legal Chess Diagrams Using Monte Carlo Simulations}
\author{Sven Hans}
\date{October 2025}

\begin{document}
\maketitle

\begin{abstract}
Determining the exact number of legal chess diagrams remains an open problem. This paper presents a statistical approach based on material classes and Monte Carlo simulations to estimate the number of legal diagrams. We define material classes, explain their enumeration, and demonstrate how random sampling can yield approximate legal ratios for each class.
\end{abstract}

\section{Introduction and Motivation}

\subsection{Scientific Overview}

Estimating the total number of possible chess diagrams or positions has long been a topic in chess mathematics. Claude Shannon \cite{shannon1950} gave an early rough estimate of the number of possible chess positions. His statement reads:

\begin{quote}
``The number of possible positions, of the general order of 
$64! / (32! \, 8!^2 \, 2!^6)$, or roughly $10^{43}$.''
\end{quote}

This statement originates from Shannon’s foundational 1950 paper. His work focused primarily on the complexity of chess as a game rather than the enumeration of strictly legal positions. 
Modern studies like \cite{chinchalker1996} found an upper bound for the maximum legal chess positions by basic combinatoric and applying restrictions based on chess rules. In 2015 Steinerberger \cite{steinnerberger2015} proposed an upper bound for legal diagrams by incorporating the color-lock restriction of bishop. He additionally conjectures that an upper bound of $\times10^{35}$ for legal diagrams, if one could incorporate only legal pawn structures. Additionally he suggests a Monte Carlo approach to find a tighter upper bound.
2021 Tromp \cite{tromp2021b} implemented a Monte Carlo approach and estimated with 1 Mio samples the upper bound of legal positions (he calls them Ur positions) to $\times10^{44}$. It is not clear to the author, if the approach is restricted to games without promotions or not. 
Due to the immense combinatorial explosion, exact enumeration remains computationally infeasible. Incorporating all chess rules in a combinatoric deduction of the upper count of legal diagrams or positions seems to be way too complex. Monte Carlo sampling is currently the most practical approach.

\subsection{Motivation}

Despite decades of research, no exact enumeration of legal chess diagrams or positions exists as of 2025. 
The challenge primarily comes from two aspects: the complex verification of legality 
(ensuring positions obey chess rules such as valid pawn placement, single kings per side, 
castling and en passant rights, and promotions) and reachability 
(ensuring the position can arise from the initial setup by a series of legal moves). 

In addition, many authors apply different restrictions and assumptions when defining or estimating 
the number of chess positions or diagrams. These include whether promotions are allowed, whether illegal pawn 
placements are excluded, or whether only reachable positions are considered. Such differences lead 
to significantly varying numerical estimates. Table~\ref{tab:definitions} summarizes typical definitions 
and includes representative values from the literature.

\begin{table}[h!]
\centering
\renewcommand{\arraystretch}{1.3}
\begin{tabular}{p{3.8cm} p{6.2cm} p{4.9cm}}
\toprule
\textbf{Definition} & \textbf{Description} & \textbf{Upper bounds} \\
\midrule
All diagrams & All possible placements of pieces on the board, ignoring any legality constraints. 
& $\sim 10^{43}$ \cite{shannon1950} \\[0.3em]

Legal diagrams & Diagrams satisfying chess rules (e.g., exactly one king per side, no pawns on rank~1 or~8) without promotions 
& $\approx 1,5\times10^{40}$ \cite{steinerberger2015} $\approx 2.9\times10^{39}$ \cite{tromp2021b} 4\times 10^{37} \cite{gourion2021} \\[0.3em]

Legal positions & Legal diagrams with additional information including side to move, castling rights and en~passant possibilities. 
& $\approx (1.78\pm0.04)\times10^{46}$ \cite{chinchalkar1996} , $\approx (4.8\pm0.04)\times10^{44}$ \cite{tromp2021b} \\[0.3em]

Reachable positions & Legal positions that can be reached from the initial setup through valid move sequences. 
&  \\[0.3em]

Legal with promotions & Legal positions including all possible promotions (e.g., multiple queens). 
& \\
\bottomrule
\end{tabular}
\caption{Different definitions of chess \textit{diagrams} and \textit{positions} with corresponding approximate estimates from the literature.}
\label{tab:definitions}
\end{table}

In this paper we try a similar approach like Tromp \cite{tromp2021b}, however we will count legal diagrams without promotions, not positions. In contrary to Trump's work, we will sample multiple sub spaces defined by single material classes with Monte Carlo and not the full space of diagrams at once.


\section{Methods}

\subsection{Material Classes}

We define a \emph{material class} as a set of specific chess pieces. Each material class specifies the number of kings $K$, queens $Q$, rooks $R$, bishops $B$, knights $K$, and pawns $P$ present for white and black.

\subsubsection{Analytical Calculation}

The number of different material classes can be computed as follows: 
For a single side, the total number of possible piece count combinations with no promotions (limits: $K = 1, Q \le 1, R \le 2, B \le 2, N \le 2, P \le 8$) is:
\[
\text{num_combinations} = (K) × (Q + 1) \cross (R + 1) \cross (B + 1) \cross (N + 1) \cross (P + 1)

\]
Substituting the limits:
\[
1 \cdot (1 + 1) \cdot (2 + 1) \cdot (2 + 1) \cdot (2 + 1) \cdot (8 + 1) = 486.
\]

Since white’s and black’s piece counts are independent, the total number of

combined material classes is:
\[
486^2 = 236{,}196.
\]

\subsubsection{Python Implementation}

The following Python code enumerates all material classes, checking piece
count constraints and total pieces:

\begin{lstlisting}[language=Python, caption={Enumeration of all material classes in chess}, label={lst:material-classes}]
import itertools

limits = {"K": 1, "Q": 1, "R": 2, "B": 2, "N": 2, "P": 8}

def all_side_materials():
    side_classes = []
    for q in range(limits["Q"] + 1):
        for r in range(limits["R"] + 1):
            for b in range(limits["B"] + 1):
                for n in range(limits["N"] + 1):
                    for p in range(limits["P"] + 1):
                        total_pieces = 1 + q + r + b + n + p  # Including 1 king always
                        if total_pieces <= 16:  # Max pieces per side
                            side_classes.append({
                                "K": 1, "Q": q, "R": r, "B": b, "N": n, "P": p
                            })
    return side_classes

white_materials = all_side_materials()
black_materials = all_side_materials()

material_classes = [
    (w, b) for w, b in itertools.product(white_materials, black_materials)
]

print(len(material_classes))  # 236196
\end{lstlisting}

\noindent
\textbf{Note:} The check for total pieces per side $\leq 16$ (half the board) and total
pieces $\leq 32$ can be added for enhanced correctness. When including promotions, 
these limits on individual pieces can be increased, e.g., queen count
from 0 to 9, representing promotions to multiple queens.

\begin{figure}[h!]
  \centering
  \includegraphics[width=0.7\textwidth]{material_class_histogram.png}
  \caption{Distribution of the count of different material classes for chess pieces count of 2-32}
  \label{fig:material_class_hist}
\end{figure}
In figure \{fig:material_class_hist} the distribution of different material classes over the number of chess pieces is shown.
\subsection{Diagram Enumeration Within a Material Class}

The total number of ways to distribute the given pieces of a material class over 64 squares (ignoring legality) can be expressed combinatorially as:

\[
\text{Diagrams} = \frac{64!}{(64 - n_w - n_b)! \cdot \prod_i c_i!}
\]

where:
\begin{itemize}
\item $n_w$ and $n_b$ denote the number of white and black pieces, respectively,
\item $c_i$ denotes the count of identical pieces (e.g., two rooks, eight pawns).
\end{itemize}

\paragraph{Combinatorial Derivation.}
Let \(n_w\) denote the number of white pieces and \(n_b\) the number of black pieces, giving a total of \(n = n_w + n_b\) pieces to be placed on the 64 squares of the chessboard.  
If all pieces were distinguishable (e.g., labeled), the number of possible placements would correspond to the number of injective mappings from an \(n\)-set to the 64 squares:
\[
64 \cdot 63 \cdot 62 \cdots (64 - n + 1) = \frac{64!}{(64 - n)!}.
\]

In practice, however, many pieces are identical (e.g., the eight white pawns or two black rooks).  
Let \(c_i\) denote the count of identical pieces of type \(i\).  
Since permutations among identical pieces do not yield distinct arrangements, we must divide by the product of the corresponding factorials.  
Thus, the total number of distinct arrangements (ignoring legality) is:
\[
\text{Positions} = \frac{64!}{(64 - n)! \, \prod_i c_i!}.
\]

This formula counts purely geometric placements of the given pieces—commonly referred to as \emph{diagrams}—and ignores all legality constraints (e.g., pawns on the first or eighth rank, both kings in check, etc.).

\paragraph{Example.}
White: 1 King, 1 Queen, 3 Pawns (\(n_w = 5\)) \\
Black: 1 King, 2 Rooks (\(n_b = 3\)) \\
Total \(n = 8\).  
The factorials for identical pieces are \(1!, 1!, 3!, 1!, 2!\).  
Hence:
\[
\text{Positions} = \frac{64!}{(64 - 8)! \, 1! \, 1! \, 3! \, 1! \, 2!}
= \frac{64!}{56! \cdot 3! \cdot 2!}
= 14{,}871{,}915{,}636{,}480 \approx 1.49 \times 10^{13}.
\]

\noindent
\textbf{Note:}  
This formula counts only board \emph{diagrams}.  
A full \emph{position} in chess additionally includes meta-information such as the side to move, castling rights, and possible en-passant targets.  
Therefore, the total number of possible positions is higher, depending on the material configuration.  
Moreover, further constraints must be applied to filter only \emph{legal} positions (e.g., valid king placement, pawn locations, and consistency of castling and check states).

\subsubsection{Legal Diagram Verification Using \texttt{python-chess}}

We use the \texttt{python-chess} library~\cite{python-chess} to validate chess diagrams, ensuring that they satisfy all basic structural legality conditions. 
The validation routine checks a comprehensive set of status flags to detect inconsistencies or impossible states in a given diagram. 
Among others, the following criteria are verified automatically:

\begin{itemize}
    \item Exactly one king per side (no missing or duplicate kings)
    \item No pawns on the first or last rank
    \item Valid castling rights
    \item Valid en passant squares and consistency with pawn structure
    \item No king in check if it is that side’s turn
    \item No more than two simultaneous checks (\texttt{STATUS\_TOO\_MANY\_CHECKERS})
    \item No geometrically impossible double checks (\texttt{STATUS\_IMPOSSIBLE\_CHECK}), 
          e.g., two attacking pieces aligned on the same ray with the king,
          or en passant positions that cannot have arisen from any legal move
\end{itemize}

\noindent
If none of these error conditions are triggered, the position is marked as 
\texttt{STATUS\_VALID}, meaning it satisfies all basic requirements of internal consistency.
However, this does \emph{not} guarantee that the diagram is reachable by a legal sequence of moves from the initial position---it merely ensures that it respects the fundamental structural rules of chess.

\begin{lstlisting}[language=Python, caption={Validation of chess positions using python-chess}, label={lst:is_legal}]
import chess

def is_legal(fen):
    board = chess.Board(fen)
    return board.is_valid()
\end{lstlisting}

\noindent
\textbf{Important:} The \texttt{python-chess} library does \emph{not} check for bishop color complex legality 
(e.g., disallowing two bishops of the same color on one side in non-promotion cases). 
Such configurations can occur legally only through pawn promotion. 
Therefore, when promotions are not explicitly modeled, the valid ratios measured here 
tend to be slightly overestimated compared to stricter definitions. 
This aspect should be addressed in further work.

\subsubsection{Monte Carlo Sampling and Valid Ratio}

Random samples of positions are generated per material class. 
Let $n$ be the number of samples and $v$ the number of valid positions. 
The valid ratio is defined as:

\begin{equation}
\text{valid ratio} = \frac{v}{n}
\end{equation}

To determine the sample size $n$ needed for stable estimates, the following heuristic procedure was applied:

\begin{enumerate}
    \item Start with $n = 1000$ samples.
    \item Calculate the valid ratio.
    \item Double the sample size: $n \rightarrow 2n$, resample, and recalculate.
    \item Continue doubling until the difference between consecutive valid ratios is less than 10\%.
\end{enumerate}

\noindent
This approach stabilizes the valid ratio estimate but remains statistically \emph{ad hoc} and 
does not provide formal confidence intervals. For future improvements, formal 
statistical methods such as the computation of confidence intervals or sequential 
sampling strategies based on variance estimates would provide more rigorous 
convergence guarantees.

\section{Results}

Based on the Monte Carlo sampling approach described above, the total number of \emph{legal chess diagrams} (i.e., piece arrangements satisfying the structural rules of chess but not necessarily reachable by legal play) was estimated.

For each material class $m$, the total number of diagrams was computed combinatorially, and multiplied by the empirically estimated \emph{valid ratio}~$r_m$ obtained from the Monte Carlo simulations.  
The overall estimate across all material classes is therefore given by:

\[
N_{\text{legal}} = \sum_{m \in \mathcal{M}} r_m \cdot N_m,
\]

where:
\begin{itemize}
    \item $\mathcal{M}$ denotes the set of all material classes (possible distributions of pieces by type and color),
    \item $N_m$ is the total number of diagrams for material class~$m$,
    \item $r_m$ is the fraction of those diagrams that are structurally legal according to \texttt{python-chess}.
\end{itemize}

Using this procedure, the estimated upper bound for the total number of legal diagrams is:

\[
N_{\text{legal}} \approx 2.28 \times 10^{40}.
\]
To compute this bound we did calculate about 4000 samples per material class and checked the legality:In total about $1\times 10^{9}$ sample diagrams.
\noindent
When compared to the total number of all possible diagrams (i.e., all unconstrained piece placements across all material classes), the global ratio of legal to total diagrams was found to be approximately:

\[
\frac{N_{\text{legal}}}{N_{\text{all}}} \approx 1.92 \times 10^{-3}.
\]

This indicates that fewer than one in every five hundred random piece configurations satisfies even the basic structural rules of chess, stated above.
The valid ratio~$r_m$ is, however, \emph{not constant} across material classes: positions with a small number of pieces (e.g., king and pawns only) or heavily asymmetrical material distributions exhibit a higher probability of structural legality, while classes with many pieces or numerous pawns show significantly lower ratios due to increased likelihood of overlapping or invalid pawn placements.

These results represent an upper bound, as the legality checks performed do not include color-complex restrictions for bishops or reachability constraints based on move sequences. The legality check does also not include unreachable pawn patterns, like more than 4 prawns in one column or unreachable side-pawn patterns \cite{mcdonagh2022}.

\begin{figure}[h!]
  \centering
  \includegraphics[width=0.6\textwidth]{positions_per_class.png}
  \caption{Count of diagrams and legal diagrams over the rank of each material class. The rank of a classes is defined by sorting by the number of all diagrams of the classes.}
  \label{fig:diagrams_per_class}
\end{figure}
In \ref{fig:diagrams_per_class} one may observe, that only very few of material classes have a huge impact on the total number of possible (legal) diagrams. Therefore it would be sufficient to only analyze these classes to come up with a good approximation of the stated upper bound for legal diagrams. The noise in the legal diagrams line, is probably due to the statistical approach and a too small sample size. However, we also expect a variation as the true legal ratio may fluctuate between material classes with adjacent rank.

\begin{figure}[h!]
  \centering
  \includegraphics[width=0.6\textwidth]{valid_ratio_per_class.png}
  \caption{Legal ratio over the rank of material classes.}
  \label{fig:legal_ratio_by_class}
\end{figure}

The figure \ref{fig:diagrams_per_class} shows the legal ratio over all classes, with classes sorted in the same way, as in figure \ref{fig:diagrams_per_class}. One may observe a noisy variation of the legal ratio in dependence of the material class. There is also a systematic drift, with low ratio for classes with high rank. Meaning the fraction of illegal diagrams is higher for classes with large counts of different diagrams.


\begin{figure}[h!]
  \centering
  \includegraphics[width=0.6\textwidth]{valid_ratio_per_num_pieces.png}
  \caption{Average Legal ratio over number of chess pieces}
  \label{fig:legal_ratio_by_pieces}
\end{figure}
Figure \ref{fig:legal_ratio_by_pieces} shows the average legal ratio over the number of chess pieces. One observes, that the lowest fraction of legal diagrams can be found for 32 pieces. This indicates again, that for solving the problem of finding an upper bound for legal diagrams or positions, analyzing games with high number of pieces is most important.
\begin{figure}[h!]
  \centering
  \includegraphics[width=0.6\textwidth]{positions_per_num_pieces.png}
  \caption{Average (legal) diagrams ratio over number of chess pieces}
  \label{fig:diagrams_by_pieces}
\end{figure}

Finally figure \ref{fig:diagrams_by_pieces} shows the number of total and legal diagrams over the number of chess pieces. Interestingly we find the maximum of legal diagrams for 31 pieces and not 32, like for the total number of diagrams.



\section{References}

\begin{thebibliography}{9}

\bibitem{shannon1950}
C. E. Shannon, \emph{Programming a Computer for Playing Chess}, 
\textit{Philosophical Magazine}, Vol.~41, pp.~256–275, 1950.

\bibitem{chinchalkar1996}
S.,S. Chinchalkar, \emph{An Upper Bound for the Number of Reachable Positions},
\textit{ICCA Journal}, Vol. 19, No. 3 (Sept. 1996), pp. 181–183. DOI:10.3233/ICG-1996-19305.

\bibitem{steinerberger2015}
S.,Steinerberger, \emph{On the number of positions in chess without promotion},
\textit{International Journal of Game Theory}, Vol.,44, No.,3 (Aug. 2015), pp.,761-767. DOI:10.1007/s00182-014-0453-7. 

\bibitem{tromp2021b}
J. Tromp, \emph{Chess Position Ranking}, GitHub repository, Dec.~2021. 
\url{https://github.com/tromp/ChessPositionRanking}

\bibitem{python-chess}
N. Fiekas, “python-chess: A Chess Library for Python,” 
\url{https://python-chess.readthedocs.io/}

\bibitem{gourion2021}
D. Gourion, \emph{An Upper Bound for the Number of Chess Diagrams Without Promotion}, 
arXiv preprint \textit{arXiv:2112.09386}, 2021.

\bibitem{mcdonagh2022}
C. McDonagh, \emph{Counting Unreachable Single-Side Pawn Diagrams with Limitless Captures}, 
\textit{arXiv preprint} arXiv:2202.00428v1, 2022.

\end{thebibliography}

\end{document}
